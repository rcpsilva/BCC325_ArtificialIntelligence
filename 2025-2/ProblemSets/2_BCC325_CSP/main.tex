\documentclass{article}
\usepackage[utf8]{inputenc}
\usepackage[margin=1.2in]{geometry}
\usepackage{hyperref}

\usepackage{tikz}
\usetikzlibrary{positioning}

\usepackage{natbib}
\usepackage{graphicx}
\usepackage{amsmath}

\title{\vspace{-2 cm}Universidade Federal de Ouro Preto \\ BCC 325 - Inteligência Artificial \\ Raciocínio com Restrições}
\author{Prof. Rodrigo Silva}
\date{}

\begin{document}

\maketitle

\section{Leitura}

\begin{itemize}
    \item Capítulo 4 do livro \textit{Artificial Intelligence: Foundations of Computational Agents, 3rd Edition}, de David L. Poole e Alan K. Mackworth. Disponível em \textit{https://artint.info/}
\end{itemize}

\section{Questões}

\begin{enumerate}
    \item (Seção 4.1.1) O que é uma variável em um problema de satisfação de restrições (CSP)? Diferencie variáveis discretas de variáveis contínuas.

    \item (Seção 4.1.1) O que é uma atribuição total de variáveis? Quantas atribuições totais existem para um CSP com $n$ variáveis, cada uma com domínio de tamanho $d$?

    \item (Seção 4.1.2) Defina e exemplifique:
    \begin{itemize}
        \item (a) Restrição unária;
        \item (b) Restrição binária;
        \item (c) Restrição ternária;
        \item (d) Definição intencional e extensional de uma restrição.
    \end{itemize}

    \item (Seção 4.2) Considere o CSP com variáveis $A, B, C \in \{1, 2, 3, 4\}$ e restrições $A < B$ e $B < C$:
    \begin{itemize}
        \item (a) Quantas atribuições totais existem?
        \item (b) Quantas dessas atribuições satisfazem todas as restrições?
        \item (c) Esboce a árvore de busca até o nível em que ocorre a primeira poda por violação de restrição.
    \end{itemize}

    \item (Seção 4.1.1) Qual é a vantagem de modelar estados por meio de variáveis em vez de enumerar todos os estados explicitamente?

    \item (Seção 4.2) Explique o funcionamento do algoritmo \texttt{DFS\_solver} da Figura 4.1. Por que a verificação de restrições em atribuições parciais pode reduzir significativamente o espaço de busca?

    \item (Seção 4.1.3) Considere o seguinte problema de agendamento de entregas com as variáveis $A, B, C, D, E$, todas com domínio $\{1, 2, 3, 4\}$ e as seguintes restrições:
    \[
    \{(B \neq 3), (C \neq 2), (A \neq B), (B \neq C), (C < D), (A = D), (E < A), (E < B), (E < C), (E < D), (B \neq D)\}
    \]

    \begin{itemize}
        \item (a) Encontre uma solução que satisfaça todas as restrições;
        \item (b) Justifique quais atribuições puderam ser descartadas por poda antes de explorar totalmente a árvore de busca.
    \end{itemize}

    \item (Reflexiva) Descreva uma situação do mundo real (diferente dos exemplos dados no livro) que pode ser modelada como um CSP. Especifique as variáveis, seus domínios e pelo menos duas restrições.
    
    \item (Seção 4.3) Consistência de arco e GAC:
    \begin{itemize}
        \item (a) Defina o que significa um arco $\langle X, c\rangle$ ser \emph{consistente de arco} (arc-consistent) em relação ao dicionário de domínios \texttt{dom}.
        \item (b) O que significa uma rede de restrições ser \emph{arc-consistent}?
        \item (c) Descreva os três possíveis estados em que a rede pode se encontrar após a execução do algoritmo \texttt{GAC} (Figura 4.4): comente o que cada caso implica sobre a existência e unicidade de soluções.
    \end{itemize}

    \item (Seção 4.3, Exemplo 4.14/4.18) Reconsidere o CSP da Questão 4, com variáveis $A,B,C \in \{1,2,3,4\}$ e restrições $A < B$ e $B < C$.
    \begin{itemize}
        \item (a) Desenhe a rede de restrições (constraint network) correspondente, indicando nós de variáveis, nós de restrições e arcos $\langle X, c\rangle$.
        \item (b) Considere o conjunto inicial $to\_do$ contendo todos os arcos da rede. Aplique manualmente o algoritmo \texttt{GAC} (Figura 4.4), apresentando uma tabela com:
        \begin{itemize}
            \item o arco selecionado em cada passo;
            \item se houve ou não redução de domínio;
            \item quais novos arcos foram adicionados a $to\_do$.
        \end{itemize}
        \item (c) Indique os domínios finais de $A$, $B$ e $C$ após o término do algoritmo. Compare com os domínios iniciais e discuta em que sentido o problema foi “simplificado” para a busca em profundidade.
    \end{itemize}

    \item (Seção 4.3, Exemplo 4.20) Considere o CSP com variáveis $A$, $B$ e $C$, cada uma com domínio $\{1,2,3,4\}$, e restrições $A = B$, $B = C$ e $A \neq C$.
    \begin{itemize}
        \item (a) Mostre que essa rede é arc-consistent, isto é, que nenhum domínio pode ser reduzido usando apenas uma restrição por vez.
        \item (b) Mostre que, apesar disso, o CSP não possui solução (não há atribuição que satisfaça todas as restrições simultaneamente).
        \item (c) Explique por que esse exemplo mostra que a consistência de arco, por si só, não é suficiente para decidir a satisfatibilidade de um CSP em geral. Comente brevemente como técnicas de consistência de ordem superior (por exemplo, path consistency) podem ajudar nesse tipo de situação.
    \end{itemize}

\end{enumerate}

\section{Problema Prático}

\begin{enumerate}
    \item \textbf{Backtracking para Sudoku e análise experimental}

    Considere o problema de resolver jogos de Sudoku $9\times 9$ usando backtracking.

    \begin{enumerate}
        \item \textbf{Implementação do resolvedor}

        Implemente um algoritmo de \emph{backtracking} para resolver instâncias de Sudoku. Sua implementação deve:

        \begin{itemize}
            \item representar o tabuleiro como uma matriz $9\times 9$ (ou estrutura equivalente);
            \item escolher, em cada passo, uma célula ainda não preenchida;
            \item testar valores possíveis naquela célula, verificando se a atribuição é consistente com as regras do Sudoku (linha, coluna e subgrade $3\times 3$);
            \item retroceder (backtrack) quando não houver valor possível para a próxima célula.
        \end{itemize}

        \item \textbf{Contagem de atribuições avaliadas}

        Modifique seu código para contar o \emph{número de atribuições avaliadas até que a primeira solução seja encontrada}. Para isso:

        \begin{itemize}
            \item considere como \emph{atribuição avaliada} cada tentativa de atribuir um valor a uma célula ainda não preenchida (mesmo que essa tentativa venha a ser rejeitada logo em seguida por violar alguma restrição);
            \item incremente um contador toda vez que o algoritmo tentar colocar um valor em uma célula durante a busca.
        \end{itemize}

        Ao final da execução, o algoritmo deve imprimir (ou retornar) o número total de atribuições avaliadas para resolver aquela instância.

        \item \textbf{Seleção das instâncias}

        Escolha, em fontes confiáveis na internet (por exemplo, sites de Sudoku):

        \begin{itemize}
            \item 5 instâncias classificadas como “fáceis”;
            \item 5 instâncias classificadas como “difíceis” (por exemplo, “hard”, “expert” ou equivalente).
        \end{itemize}

        Para cada instância, registre:

        \begin{itemize}
            \item o identificador ou link da fonte;
            \item a classificação de dificuldade fornecida pelo site.
        \end{itemize}

        \item \textbf{Experimento}

        Para cada uma das 10 instâncias (5 fáceis e 5 difíceis):

        \begin{itemize}
            \item execute o seu algoritmo de backtracking;
            \item registre o número de atribuições avaliadas até a primeira solução ser encontrada.
        \end{itemize}

        Organize os resultados em uma tabela, separando claramente os grupos de Sudokus fáceis e difíceis.

        \item \textbf{Análise estatística}

        \begin{itemize}
            \item Calcule a \textbf{média} e o \textbf{desvio padrão} do número de atribuições avaliadas para o conjunto de problemas fáceis.
            \item Calcule a \textbf{média} e o \textbf{desvio padrão} do número de atribuições avaliadas para o conjunto de problemas difíceis.
        \end{itemize}

        Apresente os resultados (por exemplo, em uma pequena tabela) e escreva um parágrafo discutindo:

        \begin{itemize}
            \item se, no seu experimento, os Sudokus difíceis exigiram mais atribuições avaliadas que os fáceis;
            \item possíveis razões para diferenças observadas (por exemplo, estrutura dos puzzles, ordem de escolha de variáveis, heurísticas usadas ou não, etc.).
        \end{itemize}

    \end{enumerate}

\end{enumerate}


\end{document}
