\documentclass{article}
\usepackage[utf8]{inputenc}
\usepackage[margin=1.2in]{geometry}
\usepackage{hyperref}

\usepackage{tikz}
\usetikzlibrary{positioning}

\usepackage{natbib}
\usepackage{graphicx}
\usepackage{amsmath}

\title{\vspace{-2 cm}Universidade Federal de Ouro Preto \\ BCC 325 - Inteligência Artificial \\ Raciocínio com Restrições}
\author{Prof. Rodrigo Silva}
\date{}

\begin{document}

\maketitle

\section{Leitura}

\begin{itemize}
    \item Capítulo 4 do livro \textit{Artificial Intelligence: Foundations of Computational Agents, 3rd Edition}, de David L. Poole e Alan K. Mackworth. Disponível em \textit{https://artint.info/}
\end{itemize}

\section{Questões}

\begin{enumerate}
    \item (Seção 4.1.1) O que é uma variável em um problema de satisfação de restrições (CSP)? Diferencie variáveis discretas de variáveis contínuas.

    \item (Seção 4.1.1) O que é uma atribuição total de variáveis? Quantas atribuições totais existem para um CSP com $n$ variáveis, cada uma com domínio de tamanho $d$?

    \item (Seção 4.1.2) Defina e exemplifique:
    \begin{itemize}
        \item (a) Restrição unária;
        \item (b) Restrição binária;
        \item (c) Restrição ternária;
        \item (d) Definição intencional e extensional de uma restrição.
    \end{itemize}

    \item (Seção 4.2) Considere o CSP com variáveis $A, B, C \in \{1, 2, 3, 4\}$ e restrições $A < B$ e $B < C$:
    \begin{itemize}
        \item (a) Quantas atribuições totais existem?
        \item (b) Quantas dessas atribuições satisfazem todas as restrições?
        \item (c) Esboce a árvore de busca até o nível em que ocorre a primeira poda por violação de restrição.
    \end{itemize}

    \item (Seção 4.1.1) Qual é a vantagem de modelar estados por meio de variáveis em vez de enumerar todos os estados explicitamente?

    \item (Seção 4.2) Explique o funcionamento do algoritmo \texttt{DFS\_solver} da Figura 4.1. Por que a verificação de restrições em atribuições parciais pode reduzir significativamente o espaço de busca?

    \item (Seção 4.1.3) Considere o seguinte problema de agendamento de entregas com as variáveis $A, B, C, D, E$, todas com domínio $\{1, 2, 3, 4\}$ e as seguintes restrições:
    \[
    \{(B \neq 3), (C \neq 2), (A \neq B), (B \neq C), (C < D), (A = D), (E < A), (E < B), (E < C), (E < D), (B \neq D)\}
    \]

    \begin{itemize}
        \item (a) Encontre uma solução que satisfaça todas as restrições;
        \item (b) Justifique quais atribuições puderam ser descartadas por poda antes de explorar totalmente a árvore de busca.
    \end{itemize}

    \item (Reflexiva) Descreva uma situação do mundo real (diferente dos exemplos dados no livro) que pode ser modelada como um CSP. Especifique as variáveis, seus domínios e pelo menos duas restrições.

\end{enumerate}

\section{Problemas Práticos}

\begin{enumerate}
    \item Implemente o algoritmo A* para o problema do labirinto. 
    \item Considere o seguinte problema de alocação de salas:

    Uma universidade precisa alocar um conjunto de disciplinas em salas e horários disponíveis, respeitando restrições como:
    \begin{itemize}
        \item Uma sala só pode ter uma aula por horário;
        \item Um professor só pode dar uma aula por vez;
        \item A capacidade da sala deve ser suficiente para o número de alunos da disciplina;
        \item Algumas disciplinas exigem salas específicas (por exemplo, laboratórios).
    \end{itemize}

    \begin{enumerate}
        \item Modele este problema como um problema de satisfação de restrições (CSP), identificando:
        \begin{itemize}
            \item As variáveis do problema;
            \item Os domínios de cada variável;
            \item As restrições envolvidas.
        \end{itemize}

        \item Escreva um código baseado em busca com backtracking que resolva esse problema, seguindo a lógica dos exemplos de Sudoku ou N-Rainhas vistos em sala. Considere o exemplo abaixo como caso de teste. 


\textbf{Dados de Entrada:}
\begin{verbatim}
aulas = ['Aula1', 'Aula2', 'Aula3']

dominios = {
    'Aula1': [('Sala1', '08h'), ('Sala2', '08h'), ('Sala1', '10h')],
    'Aula2': [('Sala1', '08h'), ('Sala2', '10h')],
    'Aula3': [('Sala2', '08h'), ('Sala2', '10h'), ('Sala1', '14h')]
}

constraints = {
    'Aula1': {'professor': 'ProfA', 'alunos': 30, 'sala_requerida': None},
    'Aula2': {'professor': 'ProfB', 'alunos': 20, 'sala_requerida': 'Sala2'},
    'Aula3': {'professor': 'ProfA', 'alunos': 35, 'sala_requerida': None}
}

salas = {
    'Sala1': {'capacidade': 40},
    'Sala2': {'capacidade': 25}
}
\end{verbatim}

\textbf{Saída:}
\begin{verbatim}

[
    {'aula': 'Aula1', 'sala': 'Sala1', 'horario': '08h', 'professor': 'ProfA'},
    {'aula': 'Aula2', 'sala': 'Sala2', 'horario': '10h', 'professor': 'ProfB'},
    {'aula': 'Aula3', 'sala': 'Sala1', 'horario': '14h', 'professor': 'ProfA'}
]


\end{verbatim}
        

\end{enumerate}
\end{enumerate}

\end{document}
