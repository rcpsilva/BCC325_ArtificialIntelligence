\documentclass[9pt,a4paper]{extarticle}

% ======================
% Codificação, idioma e layout
% ======================
\usepackage[T1]{fontenc}
\usepackage[utf8]{inputenc}
\usepackage[brazil]{babel}
\usepackage{lmodern}
\usepackage{geometry}
\geometry{margin=2cm}
\usepackage{parskip}
\usepackage{microtype}

\usepackage{amsmath, amssymb, amsfonts, bm}
\usepackage{booktabs}
\usepackage{geometry}

% ======================
% Matemática e teoremas
% ======================
\usepackage{amsmath, amssymb, amsthm}
\newtheorem{definition}{Definição}

% ======================
% Listas
% ======================
\usepackage{enumitem}
\setlist{nosep}

% ======================
% Documento
% ======================
\begin{document}

\title{\vspace{-1em}Lista de Exercícios\\
Introdução ao Aprendizado de Máquina}
\date{}
\maketitle

\vspace{-1cm}

\begin{enumerate}[itemsep=0.6em]
    \item Defina cada um dos tipos de aprendizado de máquina a seguir:
    \begin{enumerate}
        \item Aprendizado supervisionado
        \item Aprendizado não supervisionado
        \item Aprendizado por reforço
        \item Aprendizado semi-supervisionado
    \end{enumerate}
    
    \item Para cada um dos tipos de aprendizado da questão anterior apresente duas aplicações. s
\end{enumerate}


\end{document}
