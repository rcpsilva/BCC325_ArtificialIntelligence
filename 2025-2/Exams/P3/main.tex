\documentclass[9pt,a4paper]{extarticle}

% ======================
% Codificação, idioma e layout
% ======================
\usepackage[T1]{fontenc}
\usepackage[utf8]{inputenc}
\usepackage[brazil]{babel}
\usepackage{lmodern}
\usepackage{geometry}
\geometry{margin=2cm}
\usepackage{parskip}
\usepackage{microtype}

% ======================
% Matemática e teoremas
% ======================
\usepackage{amsmath, amssymb, amsthm}
\newtheorem{definition}{Definição}

% ======================
% Listas
% ======================
\usepackage{enumitem}
\setlist{nosep}

% ======================
% Documento
% ======================
\begin{document}

\title{\vspace{-1em}Prova 3\\
Cláusulas Definidas Proposicionais, Horn Clauses, Diagnóstico por Consistência e Abdução}
\date{}
\maketitle

\vspace{-2cm}

\section*{Instruções gerais}
\begin{itemize}
    \item Indique claramente quaisquer suposições adicionais que você fizer.
    \item Quando for pedido para ``simular o algoritmo'', apresente os conjuntos intermediários (\emph{estados}) usados na execução (por exemplo, os conjuntos $C$ ou $G$).
    \item Nos exercícios conceituais, responda em poucas linhas, com foco na precisão.
\end{itemize}

% ==================================================
\section*{Questões}
% ==================================================

\begin{enumerate}[label=\textbf{\arabic*)}, itemsep=1.8em]

% --------------------------------------------------
\item Para cada fórmula abaixo, indique se é:
(i) uma cláusula definida (fato ou regra) \textbf{proposicional}, ou (ii) \textbf{não} é uma cláusula definida proposicional.
Justifique brevemente.

\begin{enumerate}[label=(\alph*)]
    \item $\mathit{ok\_sensor}.$
    \item $\mathit{alarm} \leftarrow \mathit{smoke} \land \mathit{heat}.$
    \item $\lnot \mathit{alarm}.$
    \item $\mathit{p} \lor \mathit{q} \lor \lnot \mathit{r}.$
    \item $\mathit{a} \land \mathit{b} \leftarrow \mathit{c}.$
    \item $\mathit{works} \leftarrow (\mathit{plugged} \lor \mathit{battery\_ok}).$
\end{enumerate}

% --------------------------------------------------
\item Converta cada cláusula definida abaixo para uma fórmula equivalente usando apenas $\lor$ e $\lnot$ (sem $\leftarrow$).

\begin{enumerate}[label=(\alph*)]
    \item $h \leftarrow a \land b.$
    \item $p \leftarrow q \land r \land s.$
    \item $t.$
\end{enumerate}

% --------------------------------------------------
\item Considere a cláusula:
\[
\mathit{safe} \leftarrow \mathit{locked} \land \mathit{alarm\_on}.
\]
Dê um exemplo de interpretação $I$ em que a cláusula é:
\begin{enumerate}[label=(\alph*)]
    \item verdadeira em $I$;
    \item falsa em $I$.
\end{enumerate}
Explique por que em cada caso.

% --------------------------------------------------
\item Considere a base de conhecimento $KB$:
\begin{align*}
    &a \leftarrow b \land c.\\
    &b \leftarrow d.\\
    &c \leftarrow e.\\
    &d.\\
    &e.\\
    &x \leftarrow a \land y.\\
    &y \leftarrow z.
\end{align*}

\begin{enumerate}[label=(\alph*)]
    \item Usando o procedimento bottom-up, construa a sequência de conjuntos $C$ até atingir ponto fixo, ou seja, um ponto em que aplicar a regra de prova não produz novos átomos.
    \item Liste os átomos derivados ao final.
    \item Justifique por que $\mathit{x}$ e $\mathit{y}$ não são derivados (se de fato não forem).
\end{enumerate}

% --------------------------------------------------
\item Usando a mesma $KB$ do exercício anterior, construa uma derivação top-down para a consulta:
\[
\mathsf{ask}\; a.
\]

% --------------------------------------------------
\item Ainda com a mesma $KB$, construa uma derivação top-down para a consulta $\mathsf{ask}\; x$ que:
\begin{itemize}
    \item em algum momento tente provar $\mathit{y}$ a partir de $\mathit{y} \leftarrow z$.
\end{itemize}
Mostre o ramo de prova e explique \textbf{explicitamente} por que ele falha.

% --------------------------------------------------
\item Considere a restrição de integridade:
\[
\mathit{false} \leftarrow \mathit{alarm} \land \mathit{quiet}.
\]
\begin{enumerate}[label=(\alph*)]
    \item Escreva uma fórmula equivalente usando apenas $\lor$ e $\lnot$ (sem $\leftarrow$ e sem $\mathit{false}$).
    \item Interprete em linguagem natural o que a restrição impõe sobre o mundo.
\end{enumerate}

% --------------------------------------------------
\item Considere a base $KB_2$:
\begin{align*}
    &\mathit{false} \leftarrow a \land b.\\
    &a \leftarrow c.\\
    &b \leftarrow d.\\
    &b \leftarrow e.
\end{align*}
Assuma o conjunto de assumíveis:
\[
A=\{c,d,e,f\}.
\]
\begin{enumerate}[label=(\alph*)]
    \item Mostre que $\{c,d\}$ é um conflito.
    \item Mostre que $\{c,e\}$ é um conflito.
    \item Dê um exemplo de conflito que não seja mínimo e explique por que não é mínimo.
\end{enumerate}

% --------------------------------------------------
\item No diagnóstico por consistência, considere um conjunto de \emph{assumíveis} $A$.
Um \emph{diagnóstico} é um subconjunto $D \subseteq A$ de assumíveis supostos falhos tal que, ao desconsiderar os elementos de $D$, a inconsistência do sistema é eliminada.
Equivalentemente, um diagnóstico deve \emph{intersectar todo conflito mínimo} (isto é, ser um \emph{hitting set} dos conflitos mínimos).

Suponha que, para um sistema, os conflitos mínimos (sobre assumíveis) são:
\[
C_1=\{p,q,r\}, \qquad C_2=\{q,s\}.
\]

\begin{enumerate}[label=(\alph*)]
    \item Dê dois exemplos de diagnósticos (conjuntos de assumíveis) que intersectam ambos os conflitos.
    \item Dê um diagnóstico mínimo e um que não seja mínimo.
    \item Em linguagem natural, explique o que significa ``diagnóstico mínimo'' neste contexto.
\end{enumerate}


% --------------------------------------------------
\item Considere o domínio:
\begin{align*}
    \mathit{bronchitis} &\leftarrow \mathit{influenza}.\\
    \mathit{bronchitis} &\leftarrow \mathit{smokes}.\\
    \mathit{coughing} &\leftarrow \mathit{bronchitis}.\\
    \mathit{fever} &\leftarrow \mathit{influenza}.\\
    \mathit{fever} &\leftarrow \mathit{infection}.\\
    \mathit{false} &\leftarrow \mathit{smokes} \land \mathit{nonsmoker}.
\end{align*}
Assumíveis:
\[
A=\{\mathit{smokes},\mathit{nonsmoker},\mathit{influenza},\mathit{infection}\}.
\]
Para cada observação abaixo, liste \textbf{pelo menos uma} explicação mínima (conjunto de assumíveis) e justifique a minimalidade.

\begin{enumerate}[label=(\alph*)]
    \item $\mathit{coughing}$.
    \item $\mathit{coughing} \land \mathit{fever}$.
    \item $\mathit{fever} \land \mathit{nonsmoker}$.
\end{enumerate}

\end{enumerate}

\end{document}
