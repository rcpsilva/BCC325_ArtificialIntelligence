\documentclass[12pt]{article}
\usepackage[utf8]{inputenc}
\usepackage{graphicx}
\usepackage[brazil]{babel}
\usepackage{amsmath, amssymb}
\usepackage{geometry}
\geometry{a4paper, margin=2.5cm}
\title{BCC325 – Prova 1 \\ Inteligência Artificial}
\author{Universidade Federal de Ouro Preto}
\date{}

\begin{document}

\maketitle

\vspace{0.5cm}

\textbf{Instruções:} Responda às questões a seguir com clareza e objetividade. Justifique todas as suas respostas sempre que possível. 

\begin{enumerate}

\item (2,0 pts) No contexto de busca em espaços de estados, explique:
\begin{enumerate}
    \item O que caracteriza um problema de busca.
    \item Quais são os componentes fundamentais de um problema de busca.
\end{enumerate}

\item (2,0 pts) Considere o problema de encontrar o caminho da posição $(0,0)$ à posição $(3,3)$ em um gride 4x4. Apresente o grafo referente a este problema. Considere que não é possível voltar a uma posição já visitada.

\item (2,0 pts) Apresente o algoritmo genérico de busca e explique como ele pode ser adaptado para busca em profundidade e busca em largura. Mostre as diferenças principais entre essas duas estratégias quanto a:
\begin{itemize}
    \item estrutura de dados usada;
    \item tipo de solução encontrada;
    \item complexidade de tempo e espaço.
\end{itemize}


\item (2,0 pts) Considere o labirinto abaixo. Apresente, passo a passo, a execução do algoritmo de busca em largura com poda de múltiplos caminhos. Início em $s$, posição $(0,0)$, e objetivo, $g$, em $(3,3)$. Represente a fronteira a cada iteração. Considere as células livres (branco) e bloqueios (preto). 

\begin{figure}[!ht]
\centering
\includegraphics[width=0.3\textwidth]{labirinto.png}
\caption{Labirinto para a busca em Profundidade}
\end{figure}

\item (1 pt) os seguintes algoritmos: Busca em Largura (BFS), Busca em Profundidade (DFS), Busca de Menor Custo Primeiro, e Busca A*.  

Para cada algoritmo, indique se ele é completo e ótimo, justificando brevemente sua resposta em função das condições necessárias sobre o espaço de estados e as funções de custo.

\medskip
\noindent\textbf{Dica:} Discuta condições como fator de ramificação finito, existência de custos positivos e admissibilidade/consistência da heurística.
\bigskip

\item (1 pt) Considere um agente que utiliza a Busca em Profundidade em um problema com:
\begin{itemize}
    \item fator de ramificação $b$;
    \item profundidade da solução $d$.
\end{itemize}

Derive a complexidade de tempo e de espaço em função de $b$ e $d$.  

\end{enumerate}

\end{document}
