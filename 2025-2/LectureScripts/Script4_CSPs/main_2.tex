\documentclass[9pt,a4paper]{extarticle}
\usepackage{parskip}

% =====================
% Codificação, idioma e layout
% =====================
\usepackage[T1]{fontenc}
\usepackage[utf8]{inputenc}
\usepackage[brazil]{babel}
\usepackage{lmodern}
\usepackage{geometry}
\geometry{margin=2cm}
\usepackage{microtype}

% =====================
% Matemática e símbolos
% =====================
\usepackage{amsmath, amssymb, amsthm}
\usepackage{mathtools}
\usepackage{bm}

% =====================
% Links
% =====================
\usepackage{hyperref}
\hypersetup{colorlinks=true,linkcolor=black,citecolor=black,urlcolor=blue}

% =====================
% Código-fonte (listings)
% =====================
\usepackage{listings}
\usepackage{xcolor}
\definecolor{codebg}{rgb}{0.96,0.96,0.96}
\definecolor{codekw}{RGB}{33,96,171}
\definecolor{codecm}{RGB}{110,110,110}
\lstset{
  backgroundcolor=\color{codebg},
  basicstyle=\ttfamily\small,
  keywordstyle=\color{codekw}\bfseries,
  commentstyle=\color{codecm}\itshape,
  stringstyle=\color{purple},
  showstringspaces=false,
  frame=single,
  frameround=tttt,
  breaklines=true
}

% =====================
% Início do documento
% =====================
\begin{document}

\title{Roteiro de Aula -- Consistência de Arcos em CSPs (Poole \& Mackworth, Cap. 4)}
\author{Prof. Rodrigo Silva}
\date{}
\maketitle

\section*{Objetivo da Aula}
Compreender o conceito de \textbf{consistência de restrições} em problemas de satisfação de restrições (CSPs), introduzir o algoritmo \textbf{GAC -- Generalized Arc Consistency}, e discutir sua aplicação prática e eficiência.

\section{Introdução}
\textbf{Problemas de Satisfação de Restrições (CSPs)} são definidos por um triplo:
\[
\langle V, D, C \rangle
\]
onde $V$ é o conjunto de variáveis, $D$ seus domínios, e $C$ o conjunto de restrições.

O método \emph{generate and test} é ineficiente, pois repete verificações. O \emph{backtracking search} melhora, mas ainda reavalia inconsistências já conhecidas. \\
\textbf{Ideia central:} eliminar valores inconsistentes dos domínios antes ou durante a busca.

\paragraph{Exemplo 4.13.} Se $A<B$ e $dom(B)=\{1,2,3,4\}$, então $A=4$ é inconsistente. Podemos eliminar 4 de $dom(A)$ antes de buscar soluções.

\section{Redes de Restrições}
Um CSP pode ser representado como uma \textbf{rede bipartida de restrições}:
\begin{itemize}
    \item Nós circulares: variáveis ($A$, $B$, $C$);
    \item Nós retangulares: restrições ($A<B$, $B<C$);
    \item Arcos: conexões $\langle X, c \rangle$ entre variáveis e restrições.
\end{itemize}

\paragraph{Exemplo 4.14.} CSP com $A,B,C \in \{1,2,3,4\}$ e restrições $A<B$, $B<C$.  
A rede contém quatro arcos:
\[
\langle A, A<B\rangle, \ \langle B, A<B\rangle, \ \langle B, B<C\rangle, \ \langle C, B<C\rangle
\]

\section{Consistência de Domínio e de Arco}
\textbf{Consistência de domínio:} toda atribuição possível de uma variável satisfaz suas restrições unárias.

\paragraph{Exemplo 4.16.} Para $B\neq 3$ e $dom(B)=\{1,2,3,4\}$, o domínio não é consistente. Removendo 3, torna-se consistente.

\textbf{Consistência de arco:} para cada $x \in dom[X]$, deve haver pelo menos uma atribuição às variáveis relacionadas que satisfaça a restrição.

\paragraph{Exemplo 4.17.} No CSP $A<B$, $B<C$, o arco $\langle A, A<B\rangle$ é inconsistente se $A=4$, pois não há valor de $B$ que satisfaça $A<B$. Assim, removemos $4$ de $dom(A)$.

\section{O Algoritmo GAC}
\textbf{Objetivo:} tornar a rede de restrições arc-consistente.

\begin{lstlisting}[language=Python, caption={Pseudocódigo do algoritmo GAC (Poole \& Mackworth, 2023)}]
procedure GAC(Vs, dom, Cs, to_do):
    while to_do not empty do
        select and remove (X, c) from to_do
        let {Y1, ... , Yk} = scope(c) \ {X}
        ND = { x $\in$ dom[X] , exists y1 ... ,yk $\in$ dom[Y1] ... dom[Yk]
                such that c(X=x,Y1=y1,...,Yk=yk) holds }
        if ND not equal dom[X] then
            dom[X] = ND
            to_do = to_do U {(Z, c_p) , {X,Z} $\subset$ scope(c_p, c_p != c, Z != X)}
    return dom
\end{lstlisting}

\paragraph{Interpretação:}
\begin{itemize}
    \item Iterativamente remove valores inconsistentes dos domínios.
    \item Se o domínio de $X$ é reduzido, outros arcos que dependem de $X$ são reavaliados.
\end{itemize}

\paragraph{Exemplo 4.18.}  
Para $A<B$ e $B<C$, inicialmente:
\[
dom(A)=dom(B)=dom(C)=\{1,2,3,4\}
\]
Após aplicar o GAC:
\[
dom(A)=\{1,2\}, \quad dom(B)=\{2,3\}, \quad dom(C)=\{3,4\}
\]

\section{Casos de Término e Interpretação}
Ao final do algoritmo:
\begin{enumerate}
    \item Algum domínio vazio $\Rightarrow$ sem solução.
    \item Todos os domínios unitários $\Rightarrow$ solução única.
    \item Domínios múltiplos não-vazios $\Rightarrow$ problema reduzido, mas requer busca.
\end{enumerate}

\paragraph{Exemplo 4.19.} Rede de agendamento: $A=4, B=2, C=3, D=4, E=1$.  
\textbf{Conclusão:} todas as variáveis possuem domínios únicos; o CSP tem solução única.

\paragraph{Exemplo 4.20.} CSP com $A=B$, $B=C$, $A\neq C$.  
Mesmo sendo arc-consistente, não há solução global.

\section{Complexidade e Extensões}
Para restrições binárias:
\[
O(c \, d^3)
\]
onde $c$ é o número de restrições e $d$ o tamanho médio dos domínios.

Espaço: $O(n\,d)$, com $n$ variáveis.

Extensões:
\begin{itemize}
    \item Domínios infinitos (restrições intensionais);
    \item \textbf{Path consistency}: analisa trios de variáveis;
    \item \textbf{k-consistency}: generalização para $k$ variáveis.
\end{itemize}

\section{Atividades}
\begin{enumerate}
    \item Construa o grafo de restrições para o CSP $X<Y$, $Y<Z$ e identifique os arcos inconsistentes.
    \item Aplique o algoritmo GAC passo a passo sobre $A<B$, $B<C$ com domínios $\{1,2,3,4\}$.
    \item Implemente em Python uma versão simples do GAC e teste-a em pequenos CSPs.
    \item Explique por que o GAC não garante a existência de solução, mesmo quando todos os arcos são consistentes.
\end{enumerate}

\section{Discussão Final}
\begin{itemize}
    \item A consistência de arcos é uma forma de \textbf{propagação de restrições}.
    \item Reduz o espaço de busca de forma sistemática, sem gerar atribuições completas.
    \item É frequentemente usada em conjunto com \textbf{busca com retrocesso}.
\end{itemize}

\paragraph{Leitura Recomendada:}
\begin{itemize}
    \item Poole, D.L. \& Mackworth, A.K. (2023). \textit{Artificial Intelligence: Foundations of Computational Agents}, 3rd ed., Cap. 4.
    \item Russell, S. \& Norvig, P. (2020). \textit{Artificial Intelligence: A Modern Approach}, 4ª ed., Seção 6.2.
\end{itemize}

\end{document}
