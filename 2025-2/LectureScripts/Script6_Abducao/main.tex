\documentclass[9pt,a4paper]{extarticle}

% ======================
% Codificação, idioma e layout
% ======================
\usepackage[T1]{fontenc}
\usepackage[utf8]{inputenc}
\usepackage[brazil]{babel}
\usepackage{lmodern}
\usepackage{geometry}
\geometry{margin=2cm}
\usepackage{parskip}
\usepackage{microtype}

% ======================
% Matemática e teoremas
% ======================
\usepackage{amsmath, amssymb, amsthm}

\newtheorem{definition}{Definição}

% ======================
% Documento
% ======================
\begin{document}

\section*{Prova por Contradição, Horn Clauses e Diagnóstico por Consistência}

% ==================================================
\section{Motivação: Provar por Contradição em Bases de Conhecimento}
% ==================================================

\paragraph{Exposição.}
\begin{itemize}
    \item Em muitas aplicações queremos saber \emph{que combinações de fatos são impossíveis}.
    \item Exemplo: no domínio elétrico, queremos proibir situações como
    \begin{quote}
        ``a mesma lâmpada está, ao mesmo tempo, \texttt{lit} e \texttt{dark}''.
    \end{quote}
    \item Para isso, precisamos representar contradições e raciocinar \emph{por absurdo}.
\end{itemize}

\paragraph{Conectar com lógica clássica.}
\begin{itemize}
    \item Prova por contradição: supomos algo, derivamos $\text{false}$, concluímos que a suposição é impossível.
    \item Em bases de conhecimento, vamos capturar isso com uma constante especial $\mathit{false}$ e com \emph{integrity constraints}.
\end{itemize}

% ==================================================
\section{Horn Clauses e Integrity Constraints}
% ==================================================

\begin{definition}[Integrity constraint]
Uma \emph{restrição de integridade} é uma cláusula da forma
\[
\mathit{false} \leftarrow a_1 \land \dots \land a_k,
\]
em que cada $a_i$ é um átomo e $\mathit{false}$ é um átomo especial, que deve ser falso em toda interpretação.
\end{definition}

\begin{definition}[Horn clause]
Uma \emph{Horn clause} é:
\begin{itemize}
    \item ou uma cláusula definida usual ($h \leftarrow a_1 \land \dots \land a_m$),
    \item ou uma restrição de integridade ($\mathit{false} \leftarrow a_1 \land \dots \land a_k$).
\end{itemize}
Ou seja, a cabeça de uma Horn clause é um átomo comum ou o átomo especial $\mathit{false}$.
\end{definition}

\subsection{Equivalência lógica}

\paragraph{Exposição.}
\begin{itemize}
    \item A integridade
    \[
    \mathit{false} \leftarrow a_1 \land \dots \land a_k
    \]
    é logicamente equivalente à cláusula
    \[
    \neg a_1 \lor \dots \lor \neg a_k.
    \]
    \item Interpretação: ``pelo menos um dos $a_i$ tem que ser falso''.
    \item Diferentemente de uma base apenas com cláusulas definidas, uma KB com Horn clauses pode \emph{implicar negações} de átomos.
\end{itemize}

\subsection{Exemplo simples: $KB_1$ (Exemplo 5.17)}

Considere:
\begin{align*}
\mathit{false} &\leftarrow a \land b.\\
a &\leftarrow c.\\
b &\leftarrow c.
\end{align*}

\paragraph{Raciocínio em sala.}
\begin{itemize}
    \item Suponha que $c$ seja verdadeiro em um modelo $I$.
    \item Então $a$ e $b$ também são verdadeiros em $I$ (pelas duas regras).
    \item Como $\mathit{false}$ deve ser sempre falso, a cláusula
    \[
       \mathit{false} \leftarrow a \land b
    \]
    fica violada em $I$ (corpo verdadeiro, cabeça falsa) $\Rightarrow$ $I$ não é modelo.
    \item Logo, \textbf{não} existe modelo de $KB_1$ em que $c$ seja verdadeiro.  
    Conclusão:
    \[
       KB_1 \models \neg c.
    \]
\end{itemize}

\subsection{Exemplo: disjunções de negações (Exemplo 5.18)}

Considere:
\begin{align*}
\mathit{false} &\leftarrow a \land b.\\
a &\leftarrow c.\\
b &\leftarrow d.\\
b &\leftarrow e.
\end{align*}

\paragraph{Discussão.}
\begin{itemize}
    \item Se $c$ e $d$ fossem ambos verdadeiros em um modelo $I$, então $a$ e $b$ seriam verdadeiros, violando a integridade. Logo, em todo modelo:
    \[
        KB_2 \models \neg c \lor \neg d.
    \]
    \item Pelo mesmo raciocínio:
    \[
        KB_2 \models \neg c \lor \neg e.
    \]
    \item Isto mostra que, mesmo sem escrever disjunções e negações na entrada, conseguimos derivar sentenças desse tipo.
\end{itemize}

\subsection{Satisfatibilidade e inconsitência}

\paragraph{Pontos importantes.}
\begin{itemize}
    \item Um conjunto de cláusulas é \emph{insatisfatível} se não tem modelos.
    \item Ele é \emph{provavelmente inconsistente} em relação a um procedimento de prova se $\mathit{false}$ pode ser derivado.
    \item Para bases apenas com \emph{cláusulas definidas}, sempre existe modelo (por exemplo, a interpretação que faz todos os átomos verdadeiros).
    \item Já um conjunto de \emph{Horn clauses} pode ser insatisfatível, por exemplo:
    \[
       \{a,\; \mathit{false} \leftarrow a\}.
    \]
\end{itemize}

% ==================================================
\section{Assumíveis e Conflitos}
% ==================================================

\begin{definition}[Assumível]
Um \emph{assumível} (assumable) é um átomo que pode ser assumido em uma prova por contradição. Ao provar $\mathit{false}$ a partir de assumíveis, obtemos disjunções de negações desses assumíveis.
\end{definition}

\begin{definition}[Conflito]
Se $KB$ é um conjunto de Horn clauses, um \emph{conflito} de $KB$ é um conjunto de assumíveis
\[
C = \{c_1,\dots,c_r\}
\]
tal que
\[
KB \cup \{c_1,\dots,c_r\} \models \mathit{false}.
\]
Nesse caso, podemos concluir:
\[
KB \models \neg c_1 \lor \dots \lor \neg c_r.
\]
\end{definition}

\begin{definition}[Conflito mínimo]
Um conflito $C$ é \emph{mínimo} se nenhum subconjunto próprio de $C$ também é conflito.
\end{definition}

\subsection{Exemplo: conflitos em $KB_2$ (Exemplo 5.20)}

Se o conjunto de assumíveis é $\{c,d,e,f,g,h\}$, então:
\begin{itemize}
    \item $\{c,d\}$ é um conflito mínimo de $KB_2$.
    \item $\{c,e\}$ também é um conflito mínimo.
    \item $\{c,d,e,h\}$ é um conflito, mas \textbf{não} mínimo.
\end{itemize}

% ==================================================
\section{Diagnóstico Baseado em Consistência (CBD)}
% ==================================================

\paragraph{Ideia geral.}
\begin{itemize}
    \item Temos uma descrição de \emph{como o sistema deveria funcionar} (modelo nominal).
    \item Temos observações (por exemplo, lâmpadas apagadas, apesar de interruptores ligados).
    \item Fazemos suposições de normalidade (\texttt{ok\_cb1}, \texttt{ok\_s1}, \dots) e vemos quais combinações são \emph{incompatíveis} com as observações.
\end{itemize}

\paragraph{Noções-chave.}
\begin{itemize}
    \item \textbf{Assumíveis} representam componentes supostamente normais (ou eventualmente falhos).
    \item \textbf{Conflitos} são conjuntos de suposições que não podem ser todas verdadeiras.
    \item A partir dos conflitos, obtemos \emph{diagnósticos}: conjuntos de componentes que devem estar falhos para explicar tudo.
\end{itemize}

\subsection{Exemplo: circuito elétrico (Exemplos 5.21–5.22)}

\paragraph{Contexto.}
\begin{itemize}
    \item Domínio: casa com disjuntores $cb1, cb2$, chaves $s1, s2, s3$ e lâmpadas $l1, l2$.
    \item A KB contém:
    \begin{itemize}
        \item Regras que definem quando fios estão \texttt{live\_wX}.
        \item Regras que definem quando lâmpadas estão \texttt{lit\_l1}, \texttt{dark\_l1}, etc.
        \item Restrições de integridade para proibir uma lâmpada de ser, ao mesmo tempo, \texttt{lit} e \texttt{dark}:
        \[
        \mathit{false} \leftarrow \mathit{dark\_l1} \land \mathit{lit\_l1},
        \quad
        \mathit{false} \leftarrow \mathit{dark\_l2} \land \mathit{lit\_l2}.
        \]
        \item Assumíveis de normalidade:
        \[
          \mathit{ok\_cb1}, \mathit{ok\_cb2}, \mathit{ok\_s1}, \mathit{ok\_s2}, \mathit{ok\_s3}, \mathit{ok\_l1}, \mathit{ok\_l2}.
        \]
    \end{itemize}
    \item Observações:
    \[
      \mathit{up\_s1}.\;
      \mathit{up\_s2}.\;
      \mathit{up\_s3}.\;
      \mathit{dark\_l1}.\;
      \mathit{dark\_l2}.
    \]
\end{itemize}

\paragraph{Conflitos encontrados.}
O sistema encontra dois conflitos mínimos:
\[
\{ \mathit{ok\_cb1}, \mathit{ok\_s1}, \mathit{ok\_s2}, \mathit{ok\_l1} \},
\]
\[
\{ \mathit{ok\_cb1}, \mathit{ok\_s3}, \mathit{ok\_l2} \}.
\]

Interpretando:
\begin{itemize}
    \item Em qualquer modelo compatível com as observações, \emph{não} é possível que todos esses componentes estejam ok ao mesmo tempo.
    \item Em termos lógicos:
    \[
    KB \models \neg \mathit{ok\_cb1} \lor \neg \mathit{ok\_s1} \lor \neg \mathit{ok\_s2} \lor \neg \mathit{ok\_l1},
    \]
    \[
    KB \models \neg \mathit{ok\_cb1} \lor \neg \mathit{ok\_s3} \lor \neg \mathit{ok\_l2}.
    \]
\end{itemize}

\subsection{Diagnósticos mínimos (Exemplo 5.22)}

\paragraph{Ideia.}
\begin{itemize}
    \item Dado o conjunto de conflitos, um \textbf{diagnóstico} é um conjunto de assumíveis que tem pelo menos um elemento de cada conflito.
    \item Um diagnóstico \emph{mínimo} é aquele em que nenhum subconjunto também é diagnóstico.
\end{itemize}

\paragraph{Resultado do exemplo.}
No caso do circuito, os diagnósticos mínimos correspondem a:
\[
\{\mathit{ok\_cb1}\},\quad
\{\mathit{ok\_s1},\mathit{ok\_s3}\},\quad
\{\mathit{ok\_s1},\mathit{ok\_l2}\},\quad
\{\mathit{ok\_s2},\mathit{ok\_s3}\},\quad
\{\mathit{ok\_s2},\mathit{ok\_l2}\},\quad
\{\mathit{ok\_l1},\mathit{ok\_s3}\},\quad
\{\mathit{ok\_l1},\mathit{ok\_l2}\}.
\]

Em palavras: pelo menos uma dessas combinações de componentes \emph{não} está ok no mundo real.

% ==================================================
\section{Implementações com Horn Clauses e Assumíveis}
% ==================================================

\subsection{Visão geral do procedimento bottom-up (Figura 5.9)}

\paragraph{Ideia.}
\begin{itemize}
    \item Estender o algoritmo bottom-up de cláusulas definidas.
    \item Em vez de guardar apenas ``átomo é consequência'', guardamos pares
    \[
      \langle a, A \rangle
    \]
    em que $a$ é um átomo e $A$ é um conjunto de assumíveis que implicam $a$.
\end{itemize}

\paragraph{Passos.}
\begin{enumerate}
    \item Inicializar $C$ com todos assumíveis:
    \[
      C := \{ \langle a, \{a\} \rangle : a \text{ é assumível}\}.
    \]
    \item Enquanto possível, para cada cláusula
    \[
      h \leftarrow b_1 \land \dots \land b_m
    \]
    se para cada $b_i$ existe $\langle b_i, A_i \rangle \in C$, então adicionamos
    \[
      \langle h, A_1 \cup \dots \cup A_m \rangle.
    \]
    \item Quando geramos $\langle \mathit{false}, A \rangle$, o conjunto $A$ é um conflito.
\end{enumerate}

\paragraph{Refinamento.}
\begin{itemize}
    \item Podemos podar supersets: se já temos $\langle a, A_1 \rangle$ e surge $\langle a, A_2 \rangle$ com $A_1 \subset A_2$, podemos descartar o segundo.
\end{itemize}

\subsection{Visão geral do procedimento top-down (Figura 5.10)}

\paragraph{Ideia.}
\begin{itemize}
    \item Versão top-down do raciocínio com Horn clauses.
    \item Queremos provar $\mathit{false}$, começando de $G = \{\mathit{false}\}$.
\end{itemize}

\paragraph{Passos.}
\begin{enumerate}
    \item Manter um conjunto $G$ de átomos ``a provar'' que, juntos, implicam $\mathit{false}$.
    \item Enquanto existir átomo $a \in G$ que não é assumível:
    \begin{itemize}
        \item Escolher uma cláusula $a \leftarrow B$ em $KB$.
        \item Atualizar $G := (G \setminus \{a\}) \cup B$.
    \end{itemize}
    \item Quando $G$ ficar formado apenas por assumíveis, $G$ é um conflito.
\end{enumerate}

\paragraph{Observação.}
\begin{itemize}
    \item Como na derivação top-down de cláusulas definidas, escolhas diferentes de cláusulas podem levar a conflitos diferentes ou a falha.
\end{itemize}

% ==================================================
\section{Discussão Final e Exercícios}
% ==================================================

\paragraph{Pontos para discussão em sala.}
\begin{itemize}
    \item Qual a vantagem de ter explicitamente $\mathit{false}$ na linguagem?\\
    Permite escrever restrições de integridade e detectar inconsistências simplesmente tentando provar $\mathit{false}$.

    \item Por que é útil saber quais combinações de assumíveis geram contradição?\\
    Porque essas combinações indicam quais conjuntos de hipóteses não podem ser verdadeiros ao mesmo tempo, ajudando em diagnóstico, planejamento e projeto.

    \item Como conflitos se relacionam com diagnósticos mínimos?\\
    Conflitos são conjuntos de assumíveis incompatíveis; diagnósticos mínimos são conjuntos de assumíveis que intersectam todos os conflitos (têm pelo menos um elemento de cada conflito).
\end{itemize}

\paragraph{Sugestão de exercícios.}
\begin{enumerate}
    \item Considere uma KB simples com integridade:
    \[
      \mathit{false} \leftarrow \mathit{alarm} \land \mathit{quiet}.
    \]
    e regras para $\mathit{alarm}$ e $\mathit{quiet}$.  
    Proponha um conjunto de assumíveis e encontre um conflito mínimo.

    \item A partir dos conflitos do circuito elétrico, peça aos alunos que:
    \begin{itemize}
        \item listem diagnósticos mínimos;
        \item expliquem em linguagem natural o significado de cada diagnóstico.
    \end{itemize}
\end{enumerate}


\section*{Abdução e Explicações em Bases de Conhecimento}

% ==================================================
\section{Motivação: O que é Abdução?}
% ==================================================

\paragraph{Exposição.}
\begin{itemize}
    \item Enquanto \textbf{dedução} determina o que \emph{segue logicamente} de axiomas,  
    \item e \textbf{indução} infere generalizações a partir de exemplos,  
    \item a \textbf{abdução} busca hipóteses que \emph{expliquem observações}.
\end{itemize}

Exemplos introdutórios:
\begin{itemize}
    \item Observar que uma lâmpada está apagada $\Rightarrow$ considerar hipóteses (queimada, falta de energia, interruptor desligado).
    \item Em sistemas tutores, inferir o que o aluno sabe a partir de erros cometidos.
\end{itemize}

A noção foi introduzida por **Charles Peirce**.

% ==================================================
\section{Formalização da Abdução}
% ==================================================

\begin{definition}[Assumables]
Seja $A$ um conjunto de átomos que podem ser \emph{assumidos} como hipóteses.  
Estes são os blocos básicos para construir explicações.
\end{definition}

\begin{definition}[Cenário]
Dado $\langle KB, A\rangle$, um \emph{cenário} é um conjunto $H \subseteq A$ tal que  
$KB \cup H$ é satisfatível, isto é, não contém contradições.
\end{definition}

\begin{definition}[Explicação]
Uma explicação de $g$ é um conjunto $H \subseteq A$ tal que:
\[
KB \cup H \models g
\qquad\text{e}\qquad
KB \cup H \not\models \text{false}.
\]
\end{definition}

\begin{definition}[Explicação mínima]
Uma explicação $H$ é \emph{mínima} se nenhum subconjunto estrito de $H$ também explica $g$.
\end{definition}

\subsection{Comentário didático}

\begin{itemize}
    \item Observações \textbf{não são adicionadas} à KB — devem ser explicadas.
    \item Contradições são proibidas: qualquer conjunto que implique \textbf{false} é descartado.
\end{itemize}

% ==================================================
\section{Exemplo 1: Diagnóstico Médico Simples (Exemplo 5.31)}
% ==================================================

Considere a KB:

\begin{align*}
\mathit{bronchitis} &\leftarrow \mathit{influenza}.\\
\mathit{bronchitis} &\leftarrow \mathit{smokes}.\\
\mathit{coughing} &\leftarrow \mathit{bronchitis}.\\
\mathit{wheezing} &\leftarrow \mathit{bronchitis}.\\
\mathit{fever} &\leftarrow \mathit{influenza}.\\
\mathit{fever} &\leftarrow \mathit{infection}.\\
\mathit{sore\_throat} &\leftarrow \mathit{influenza}.\\
\text{false} &\leftarrow \mathit{smokes} \land \mathit{nonsmoker}.
\end{align*}

Assumables:
\[
\{\mathit{smokes},\, \mathit{nonsmoker},\, \mathit{influenza},\, \mathit{infection}\}.
\]

\subsection{Exemplo A — Observação: \texttt{wheezing}}

Para explicar $\mathit{wheezing}$:
\[
\mathit{wheezing} \leftarrow \mathit{bronchitis}.
\]
\[
\mathit{bronchitis} \leftarrow \mathit{influenza} \quad\text{ou}\quad 
\mathit{bronchitis} \leftarrow \mathit{smokes}.
\]

Explicações mínimas:
\[
\{\mathit{influenza}\}
\qquad\text{ou}\qquad
\{\mathit{smokes}\}.
\]

\subsection{Exemplo B — Observação: \texttt{wheezing $\land$ fever}}

\[
\mathit{fever} \leftarrow \mathit{influenza} \text{ ou } \mathit{infection}.
\]

Explicações mínimas:
\[
\{\mathit{influenza}\}
\quad\text{e}\quad
\{\mathit{smokes},\,\mathit{infection}\}.
\]

\subsection{Exemplo C — Observação: \texttt{wheezing $\land$ nonsmoker}}

Como assumir $\mathit{smokes}$ leva a contradição:
\[
\{\mathit{smokes},\mathit{nonsmoker}\} \models \text{false},
\]
a única explicação mínima possível é:
\[
\{\mathit{influenza},\,\mathit{nonsmoker}\}.
\]

% ==================================================
\section{Exemplo 2: Sistema de Alarme (Exemplo 5.32)}
% ==================================================

Base de conhecimento:
\begin{align*}
\mathit{alarm} &\leftarrow \mathit{tampering}.\\
\mathit{alarm} &\leftarrow \mathit{fire}.\\
\mathit{smoke} &\leftarrow \mathit{fire}.
\end{align*}

\subsection{Observação: \texttt{alarm}}

Explicações mínimas:
\[
\{\mathit{tampering}\}, \qquad \{\mathit{fire}\}.
\]

\subsection{Observação: \texttt{alarm $\land$ smoke}}

A presença de \texttt{smoke} ``explica'' o alarme:
\[
\{\mathit{fire}\}.
\]

Não há necessidade de assumir \texttt{tampering}: hipótese explicada por outra evidência.

% ==================================================
\section{Abdução vs. Diagnóstico Baseado em Consistência (CBD)}
% ==================================================

\paragraph{Principais diferenças.}

\begin{itemize}
    \item Em CBD:
    \begin{itemize}
        \item Assume-se comportamento \textbf{normal}.
        \item Observações são adicionadas à KB.
        \item Procura-se um conjunto de componentes que, se defeituosos, explique a inconsistência.
    \end{itemize}

    \item Em Abdução:
    \begin{itemize}
        \item Hipóteses incluem \textbf{normalidade e falhas}.
        \item Observações devem ser explicadas, não adicionadas.
        \item Requer modelagem mais detalhada (regra para cada forma de comportamento).
    \end{itemize}
\end{itemize}

\paragraph{Consequência.}
Abdução produz diagnósticos mais detalhados, mas exige modelagem mais rica.

% ==================================================
\section{Exemplo 3 — Diagnóstico Elétrico (Exemplo 5.33)}
% ==================================================

Regras simplificadas:
\begin{align*}
\mathit{lit\_l1} &\leftarrow \mathit{live\_w0} \land \mathit{ok\_l1}.\\
\mathit{dark\_l1} &\leftarrow \mathit{broken\_l1}.\\
\mathit{dark\_l1} &\leftarrow \mathit{dead\_w0}.\\
\text{false} &\leftarrow \mathit{ok\_l1} \land \mathit{broken\_l1}.
\end{align*}

Assumables:
\[
\{\mathit{ok\_l1},\, \mathit{broken\_l1},\, \mathit{live\_w0},\, \mathit{dead\_w0}\}.
\]

Observações como \texttt{dark\_l1} ou \texttt{lit\_l1} são explicadas por hipóteses que tornam essas regras verdadeiras.

% ==================================================
\section{Discussão Final e Exercícios}
% ==================================================

\paragraph{Perguntas para debate.}
\begin{itemize}
    \item Por que permitir hipóteses é útil, mas perigoso?
    \item O que impede que o sistema simplesmente assuma tudo?
    \item Como garantir minimalidade das explicações?
\end{itemize}

\paragraph{Exercício para casa.}
\begin{itemize}
    \item Modele um domínio simples (ex.: funcionamento de um computador) com:
    \begin{itemize}
        \item KB com Horn clauses,
        \item assumables (normais e defeituosos),
        \item observações.
    \end{itemize}
    \item Encontre explicações mínimas para duas observações distintas.
\end{itemize}

\end{document}
