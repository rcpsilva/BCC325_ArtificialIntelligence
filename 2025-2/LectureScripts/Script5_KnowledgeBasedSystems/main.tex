\documentclass[9pt,a4paper]{extarticle}

% ======================
% Codificação, idioma e layout
% ======================
\usepackage[T1]{fontenc}
\usepackage[utf8]{inputenc}
\usepackage[brazil]{babel}
\usepackage{lmodern}
\usepackage{geometry}
\geometry{margin=2cm}
\usepackage{parskip}
\usepackage{microtype}
\usepackage{amsthm}
\newtheorem{definition}{Definição}

% ======================
% Matemática e símbolos
% ======================
\usepackage{amsmath, amssymb, amsthm, bm}

% ======================
% Listas
% ======================
\usepackage{enumitem}
\setlist{nosep}

% ======================
% Documento
% ======================
\begin{document}

\section*{Cláusulas Definidas Proposicionais e Procedimentos de Prova}

\section{Cláusulas definidas proposicionais}

\subsection{Motivação e sintaxe básica (15 min)}

\paragraph{Exposição.}
\begin{itemize}
    \item Relembrar: na lógica proposicional geral podemos combinar átomos com $\land,\lor,\neg,\rightarrow$ livremente.
    \item Destacar o problema: raciocínio geral é caro; vamos restringir a linguagem para obter \emph{procedimentos eficientes}.
    \item Definir:
    \begin{itemize}
        \item Um \textbf{átomo} (ou proposição atômica) é como na lógica proposicional: $p, q, r,\dots$.
        \item Uma \textbf{cláusula definida} é uma fórmula da forma
        \[
           h \leftarrow a_1 \land \dots \land a_m
        \]
        em que $h$ é um átomo (a \textbf{cabeça}) e cada $a_i$ é um átomo (o \textbf{corpo}).
    \end{itemize}
    \item Leitura: ``$h$ se $a_1$ e \dots e $a_m$''.
    \item Caso $m>0$: chamamos de \textbf{regra}. Caso $m=0$: \textbf{fato} (ou \textbf{cláusula atômica}); escrevemos apenas $h.$
    \item Uma \textbf{base de conhecimento} (KB) é um conjunto de cláusulas definidas.
\end{itemize}

\paragraph{Atividade rápida (em plenário).}
\begin{itemize}
    \item Apresentar na lousa algumas fórmulas e pedir para a turma classificar se são ou não cláusulas definidas:
    \begin{align*}
        &\text{(i)}\quad \mathit{apple\_is\_eaten}.\\
        &\text{(ii)}\quad \mathit{apple\_is\_eaten} \leftarrow \mathit{bird\_eats\_apple}.\\
        &\text{(iii)}\quad \neg \mathit{apple\_is\_eaten}.\\
        &\text{(iv)}\quad \mathit{happy} \lor \mathit{sad} \lor \neg \mathit{alive}.\\
        &\text{(v)}\quad \mathit{sam\_is\_in\_room} \land \mathit{night\_time} \leftarrow \mathit{switch\_1\_is\_up}.
    \end{align*}
    \item Discutir por que (iii), (iv) e (v) \textbf{não} são cláusulas definidas.
\end{itemize}

\subsection{Semântica e relação com cláusulas gerais (5 min)}

\paragraph{Exposição.}
\begin{itemize}
    \item Definir: uma cláusula definida
    \[
        h \leftarrow a_1 \land \dots \land a_m
    \]
    é \textbf{falsa} em uma interpretação $I$ se todos $a_1,\dots,a_m$ são verdadeiros em $I$ e $h$ é falso em $I$; caso contrário, a cláusula é verdadeira em $I$.
    \item Mostrar a equivalência com cláusulas disjuntivas:
    \[
        h \leftarrow b \land c \land d
        \quad\equiv\quad
        h \lor \neg b \lor \neg c \lor \neg d.
    \]
    \item Ressaltar a restrição: em uma cláusula definida há \textbf{exatamente um literal positivo}. Não podemos representar $a \lor b$ nem $\neg c \lor \neg d$ como cláusulas definidas.
\end{itemize}

\paragraph{Discussão guiada.}
\begin{itemize}
    \item Perguntar: qual a vantagem de restringir a linguagem? (ganho de estrutura e de eficiência nos procedimentos de prova).
\end{itemize}

\section{Exemplo: ambiente elétrico com cláusulas definidas (20 min)}

\subsection{Escolha do nível de abstração}

\paragraph{Exposição.}
\begin{itemize}
    \item Contexto: ambiente elétrico com fios, disjuntores, interruptores e lâmpadas.
    \item Decisão de modelagem: ignorar voltagem exata, cor dos fios, detalhes físicos. Focar em:
    \begin{itemize}
        \item fios vivos: $\mathit{live\_w1}$,
        \item lâmpadas ligadas: $\mathit{lit\_l1}$,
        \item interruptores para cima/baixo: $\mathit{up\_s2}$, $\mathit{down\_s1}$,
        \item componentes funcionando: $\mathit{ok\_l1}$, $\mathit{ok\_cb1}$, etc.
    \end{itemize}
    \item Reforçar que o computador \textbf{não entende} o significado dos nomes; só manipula símbolos.
\end{itemize}

\subsection{Fatos e regras da base de conhecimento}

\paragraph{Exposição com exemplos.}
\begin{itemize}
    \item Exemplo de fatos (cláusulas atômicas):
    \begin{align*}
        &\mathit{light\_l1}. \\
        &\mathit{light\_l2}. \\
        &\mathit{ok\_l1}. \\
        &\mathit{ok\_l2}. \\
        &\mathit{ok\_cb1}. \\
        &\mathit{ok\_cb2}. \\
        &\mathit{live\_outside}. 
    \end{align*}
    \item Exemplo de regras:
    \begin{align*}
        &\mathit{live\_l1} \leftarrow \mathit{live\_w0}.\\
        &\mathit{live\_p1} \leftarrow \mathit{live\_w3}.\\
        &\mathit{live\_w0} \leftarrow \mathit{live\_w1} \land \mathit{up\_s2}.\\
        &\mathit{live\_w3} \leftarrow \mathit{live\_w5} \land \mathit{ok\_cb1}.\\
        &\mathit{live\_w0} \leftarrow \mathit{live\_w2} \land \mathit{down\_s2}.\\
        &\mathit{live\_p2} \leftarrow \mathit{live\_w6}.\\
        &\mathit{live\_w1} \leftarrow \mathit{live\_w3} \land \mathit{up\_s1}.\\
        &\mathit{live\_w6} \leftarrow \mathit{live\_w5} \land \mathit{ok\_cb2}.\\
        &\mathit{live\_w2} \leftarrow \mathit{live\_w3} \land \mathit{down\_s1}.\\
        &\mathit{live\_w5} \leftarrow \mathit{live\_outside}.\\
        &\mathit{live\_l2} \leftarrow \mathit{live\_w4}.\\
        &\mathit{lit\_l1} \leftarrow \mathit{light\_l1} \land \mathit{live\_l1} \land \mathit{ok\_l1}.\\
        &\mathit{live\_w4} \leftarrow \mathit{live\_w3} \land \mathit{up\_s3}.\\
        &\mathit{lit\_l2} \leftarrow \mathit{light\_l2} \land \mathit{live\_l2} \land \mathit{ok\_l2}.
    \end{align*}
    \item Observações do usuário (situação atual do mundo):
    \[
      \mathit{down\_s1}.\quad
      \mathit{up\_s2}.\quad
      \mathit{up\_s3}.
    \]
\end{itemize}

\paragraph{Atividade guiada.}
\begin{itemize}
    \item Em duplas, pedir que os alunos escrevam em forma de cláusula definida a ideia:
    
    ``A lâmpada $l_2$ acende quando é uma lâmpada, está viva e está funcionando.''
    
    \item Esperado:
    \[
        \mathit{lit\_l2} \leftarrow \mathit{light\_l2} \land \mathit{live\_l2} \land \mathit{ok\_l2}.
    \]
\end{itemize}

\section{Consultas e respostas (\texttt{ask}) (15 min)}

\subsection{Definição de consulta}

\paragraph{Exposição.}
\begin{itemize}
    \item Uma \textbf{consulta} tem a forma:
    \[
        \mathsf{ask}\; b.
    \]
    onde $b$ é um átomo ou uma conjunção de átomos (como o corpo de uma regra).
    \item Semântica:
    \begin{itemize}
        \item Resposta \textbf{yes}: o corpo é consequência lógica da base de conhecimento ($KB \models b$).
        \item Resposta \textbf{no}: não é possível concluir $b$ a partir do que foi fornecido na $KB$.
    \end{itemize}
    \item Enfatizar: resposta \textbf{no} \underline{não} significa necessariamente que $b$ é falso no mundo; apenas que não pode ser \emph{derivado} da $KB$.
\end{itemize}

\subsection{Exemplos do ambiente elétrico}

\paragraph{Exposição.}
\begin{itemize}
    \item Exemplos:
    \begin{align*}
        &\mathsf{ask}\; \mathit{light\_l1}.\quad \text{resposta: yes.}\\
        &\mathsf{ask}\; \mathit{light\_l6}.\quad \text{resposta: no.}\\
        &\mathsf{ask}\; \mathit{lit\_l2}.\quad \text{resposta: yes (em todos os modelos da KB).}
    \end{align*}
    \item Discutir como o usuário que conhece o domínio interpreta essas respostas.
\end{itemize}

\paragraph{Atividade curta.}
\begin{itemize}
    \item Pedir aos alunos: dado o conjunto de fatos e regras já visto, que consultas seriam interessantes para saber se o sistema está funcionando? (por exemplo, se uma chave quebrada pode explicar uma lâmpada apagada).
\end{itemize}

\section{Provas, correção e completude (5 min)}

\paragraph{Exposição.}
\begin{itemize}
    \item Definições:
    \begin{itemize}
        \item \textbf{Prova}: demonstração mecanicamente derivável de que uma proposição decorre logicamente de uma base de conhecimento.
        \item \textbf{Teorema}: proposição que admite uma prova.
        \item \textbf{Procedimento de prova}: algoritmo (possivelmente não determinístico) para derivar consequências de uma base de conhecimento.
    \end{itemize}
    \item Notação: $KB \vdash g$ significa ``$g$ pode ser provado a partir de $KB$''.
    \item \textbf{Correção (soundness)}: se $KB \vdash g$ então $KB \models g$.
    \item \textbf{Completude}: se $KB \models g$ então $KB \vdash g$.
\end{itemize}

\section{Procedimento de prova bottom-up (encadeamento para frente) (20 min)}

\subsection{Conceitos}

\begin{definition}[Interpretação]
Uma \emph{interpretação} $I$ para uma linguagem proposicional é uma função
\[
I : \mathcal{A} \to \{\text{verdadeiro},\text{falso}\},
\]
que associa a cada átomo $p \in \mathcal{A}$ um valor de verdade.  
O valor de verdade de fórmulas compostas é definido recursivamente a partir dos conectivos lógicos.
\end{definition}

\begin{definition}[Modelo]
Uma interpretação $I$ é um \emph{modelo} de uma fórmula $\varphi$ (ou de uma base de conhecimento $KB$) se $\varphi$ é verdadeira em $I$.  
Denotamos:
\[
I \models \varphi
\qquad\text{ou}\qquad
I \models KB,
\]
quando todas as fórmulas de $KB$ são verdadeiras sob $I$.  
Neste caso, dizemos que $I$ \emph{satisfaz} $\varphi$ (ou $KB$).
\end{definition}


\subsection{Ideia e algoritmo}

\paragraph{Exposição.}
\begin{itemize}
    \item Intuição: começar do que já é conhecido (fatos) e ir \emph{empilhando} novas consequências, usando um modus ponens generalizado:
    \begin{quote}
        Se $h \leftarrow a_1 \land \dots \land a_m$ está em $KB$ e todos $a_i$ já foram derivados, então podemos derivar $h$.
    \end{quote}
    \item Manter um conjunto $C$ de átomos já derivados (consequências atuais).
    \item Esboço do algoritmo:
    \begin{enumerate}
        \item Inicializar $C := \emptyset$.
        \item Repetir:
        \begin{itemize}
            \item Escolher uma cláusula $h \leftarrow a_1 \land \dots \land a_m$ em $KB$ tal que todos $a_i \in C$ e $h \not\in C$.
            \item Adicionar $h$ a $C$.
        \end{itemize}
        \item Até não ser possível selecionar mais nenhuma cláusula.
        \item Resultado: $C$ é o conjunto de todos os átomos que são consequências lógicas de $KB$.
    \end{enumerate}
\end{itemize}

\subsection{Exemplo trabalhado}

\paragraph{Exposição guiada.}
Considere a base de conhecimento:
\begin{align*}
    &a \leftarrow b \land c.\\
    &d.\\
    &b \leftarrow d \land e.\\
    &e.\\
    &b \leftarrow g \land e.\\
    &f \leftarrow a \land g.\\
    &c \leftarrow e.
\end{align*}

\begin{itemize}
    \item Construir, passo a passo, a sequência de conjuntos $C$:
    \[
    \{\} \;\to\; \{d\} \;\to\; \{d,e\} \;\to\; \{d,e,c\} \;\to\; \{d,e,c,b\} \;\to\; \{d,e,c,b,a\}.
    \]
    \item Mostrar que o algoritmo termina com $C = \{a,b,c,d,e\}$ e não deriva $f$ nem $g$.
    \item Discutir: existe um modelo de $KB$ onde $f$ e $g$ são falsos? (sim; portanto não devem ser derivados).
\end{itemize}

\subsection{Propriedades}

\paragraph{Exposição curta.}
\begin{itemize}
    \item \textbf{Correção}: todo átomo em $C$ é consequência lógica de $KB$.
    \item \textbf{Complexidade}: cada cláusula é usada no máximo uma vez $\Rightarrow$ tempo linear no tamanho da base (assumindo indexação eficiente).
    \item \textbf{Ponto fixo}: o conjunto final $C$ é um \emph{ponto fixo} do operador de derivação (aplicar a regra não muda mais $C$).
    \item \textbf{Menor modelo (minimal model)}: a interpretação que torna verdadeiros exatamente os átomos em $C$ é o modelo minimal de $KB$.
\end{itemize}

\section{Procedimento de prova top-down (SLD resolution) (20 min)}

\subsection{Ideia geral}

\paragraph{Exposição.}
\begin{itemize}
    \item Agora partimos da \textbf{consulta} e tentamos ``explicá-la'' usando as regras disponíveis.
    \item Introduzimos uma cláusula-resposta:
    \[
        \mathit{yes} \leftarrow a_1 \land \dots \land a_m
    \]
    em que $a_1 \land \dots \land a_m$ é a consulta (corpo de uma regra).
    \item Em cada passo:
    \begin{itemize}
        \item Selecionamos um átomo do corpo (subobjetivo).
        \item Escolhemos uma cláusula em $KB$ cuja cabeça seja esse átomo.
        \item Substituímos o átomo pelo corpo da cláusula escolhida (resolução).
    \end{itemize}
    \item Quando o corpo fica vazio ($\mathit{yes} \leftarrow$), temos uma prova de \textbf{sucesso}.
\end{itemize}

\subsection{Algoritmo em termos de subobjetivos}

\paragraph{Exposição em linguagem natural.}
\begin{itemize}
    \item Mantemos um conjunto (ou lista) $G$ de subobjetivos a provar.
    \item Inicialmente, $G$ contém os átomos da consulta.
    \item Repetimos:
    \begin{enumerate}
        \item Selecionar um átomo $a \in G$ (por exemplo, o da esquerda).
        \item Escolher em $KB$ uma cláusula $a \leftarrow B$.
        \item Atualizar $G := (G \setminus \{a\}) \cup B$.
    \end{enumerate}
    \item Se, em algum momento, $G = \emptyset$, retornamos \textit{yes}.
    \item Se para um átomo selecionado $a$ não há nenhuma cláusula em $KB$ com cabeça $a$, esse ramo de prova \textbf{falha}.
\end{itemize}

\subsection{Exemplo de derivação bem-sucedida e falha}

\paragraph{Exposição.}
Usando novamente:
\begin{align*}
    &a \leftarrow b \land c.\\
    &d.\\
    &b \leftarrow d \land e.\\
    &e.\\
    &b \leftarrow g \land e.\\
    &f \leftarrow a \land g.\\
    &c \leftarrow e.
\end{align*}

\begin{itemize}
    \item Consulta: $\mathsf{ask}\; a.$
    \item Derivação bem-sucedida (selecionando sempre o átomo mais à esquerda):
    \begin{align*}
        &\mathit{yes} \leftarrow a &&\text{(consulta)}\\
        &\mathit{yes} \leftarrow b \land c &&(a \leftarrow b \land c)\\
        &\mathit{yes} \leftarrow d \land e \land c &&(b \leftarrow d \land e)\\
        &\mathit{yes} \leftarrow e \land c &&(d.)\\
        &\mathit{yes} \leftarrow c &&(e.)\\
        &\mathit{yes} \leftarrow e &&(c \leftarrow e)\\
        &\mathit{yes} \leftarrow &&(e.)
    \end{align*}
    \item Derivação que falha (escolhendo a outra regra para $b$):
    \begin{align*}
        &\mathit{yes} \leftarrow a\\
        &\mathit{yes} \leftarrow b \land c\\
        &\mathit{yes} \leftarrow g \land e \land c &&(b \leftarrow g \land e)\\
        &\dots && \text{se $g$ é selecionado, não há cláusula com cabeça $g$ $\Rightarrow$ falha.}
    \end{align*}
\end{itemize}

\subsection{Grafo de busca e loops}

\paragraph{Exposição.}
\begin{itemize}
    \item A cada escolha de cláusula, criamos um novo nó em um \textbf{grafo de busca} de respostas.
    \item Um nó representa uma cláusula-resposta (ou o conjunto $G$ de subobjetivos atuais).
    \item Vizinhos: todas as possíveis resoluções do átomo selecionado com as cláusulas de $KB$ que têm essa cabeça.
    \item Nós objetivo: cláusulas da forma $\mathit{yes} \leftarrow$ (corpo vazio).
    \item Exemplo de loop (sem poda de ciclos):
    \[
      g \leftarrow a.\quad a \leftarrow b.\quad g \leftarrow c.\quad b \leftarrow a.\quad c.
    \]
    \item Bottom-up nesse exemplo alcança o ponto fixo $\{c,g\}$ e termina; top-down com busca em profundidade pode ficar ciclando.
\end{itemize}

\section{Comparação bottom-up vs top-down e fechamento (5 min)}

\paragraph{Discussão final.}
\begin{itemize}
    \item \textbf{Bottom-up}:
    \begin{itemize}
        \item Prova todos os átomos que são consequências lógicas (conjunto $C$ completo).
        \item Cada átomo é provado no máximo uma vez.
        \item Não depende da consulta; pode ser caro se só queremos responder poucas perguntas.
        \item Termina (com KB finita) e produz o modelo minimal.
    \end{itemize}
    \item \textbf{Top-down (SLD)}:
    \begin{itemize}
        \item Focado na consulta: só prova o que é relevante para responder a pergunta.
        \item Pode reprovar o mesmo átomo várias vezes (sem memorização).
        \item Pode entrar em laço infinito sem poda.
        \item Base de linguagens como Prolog (com busca em profundidade e ordem dos átomos definida pelo programador).
    \end{itemize}
\end{itemize}

\paragraph{Sugestão de tarefa para casa.}
\begin{itemize}
    \item Pedir que os alunos:
    \begin{enumerate}
        \item Modelem um mini-domínio (por exemplo, pré-requisitos de disciplinas ou hierarquia de cargos em uma empresa) usando cláusulas definidas proposicionais.
        \item Escrevam algumas consultas de interesse.
        \item Façam, no papel, pelo menos uma derivação bottom-up e uma derivação top-down para uma consulta não trivial.
    \end{enumerate}
\end{itemize}

\end{document}
