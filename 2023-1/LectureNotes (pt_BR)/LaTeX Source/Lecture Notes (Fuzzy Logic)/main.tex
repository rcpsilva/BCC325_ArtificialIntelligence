%%% Template para anotações de aula
%%% Feito por Daniel Campos com base no template de Willian Chamma que fez com base no template de  Mikhail Klassen



\documentclass[12pt,a4paper, brazil]{article}

%%%%%%% INFORMAÇÕES DO CABEÇALHO
\newcommand{\workingDate}{\textsc{\selectlanguage{portuguese}\today}}
\newcommand{\userName}{BCC740}
\newcommand{\institution}{UFOP}
\usepackage{researchdiary_png}
\usepackage{algorithm}
\usepackage[noend]{algpseudocode}
\usepackage{verbatim}
\newenvironment{metaverbatim}{\verbatim}{\endverbatim}

\begin{document}
\begin{center}
{\textbf {\huge Árvores de Decisão}}\\[5mm]
%{\large Autor: } \\[2mm]
%{\large Orientador: } \\[5mm]
\today\\[5mm] %% se quiser colocar data
\end{center}


%\section*{Resumo}

\section{Fuzzy Controller Example}


Imagine you have a basic heating system in a room, and you want to control the temperature to keep it comfortable. A fuzzy controller can be designed to adjust the heater's intensity based on the current room temperature and desired setpoint temperature.

Here are the key components of this fuzzy controller:

\begin{enumerate}
  
  \item  Inputs:

  \begin{enumerate}
    \item Error ($e$): Difference between the desired setpoint temperature and the current room temperature.
    \item Rate of Change of Error ($\Delta e$): How quickly the temperature is changing.
  \end{enumerate}

  \item Outputs:

  \begin{enumerate}
    \item Heater Intensity: A value representing how strong the heater should be, ranging from low to high.
  \end{enumerate}

  \item Fuzzy Sets:

  \begin{enumerate}
    \item   Error has fuzzy sets: Negative Big (NB), Negative Medium (NM), Zero (ZE), Positive Medium (PM), Positive Big (PB).
    \item Rate of Change of Error has fuzzy sets: Negative (N), Zero (Z), Positive (P).
    \item Heater Intensity has fuzzy sets: Low (L), Medium (M), High (H).
  \end{enumerate}

  \item Fuzzy Rules:

\begin{enumerate}
\item IF Error is NB AND $\Delta e$ is N, THEN Heater Intensity is H.
\item IF Error is NM AND $\Delta e$ is N, THEN Heater Intensity is M.
\item IF Error is ZE AND $\Delta e$ is N, THEN Heater Intensity is M.
\item IF Error is PM AND $\Delta e$ is N, THEN Heater Intensity is L.
\item IF Error is PB AND $\Delta e$ is N, THEN Heater Intensity is L.
\item IF Error is NB AND $\Delta e$ is Z, THEN Heater Intensity is H.
\item IF Error is NM AND $\Delta e$ is Z, THEN Heater Intensity is M.
\item IF Error is ZE AND $\Delta e$ is Z, THEN Heater Intensity is M.
\item IF Error is PM AND $\Delta e$ is Z, THEN Heater Intensity is M.
\item IF Error is PB AND $\Delta e$ is Z, THEN Heater Intensity is L.
\item IF Error is NB AND $\Delta e$ is P, THEN Heater Intensity is M.
\item IF Error is NM AND $\Delta e$ is P, THEN Heater Intensity is M.
\item IF Error is ZE AND $\Delta e$ is P, THEN Heater Intensity is L.
\item IF Error is PM AND $\Delta e$ is P, THEN Heater Intensity is L.
\item IF Error is PB AND $\Delta e$ is P, THEN Heater Intensity is L.
\end{enumerate}

\item Fuzzy Inference and Defuzzification:

\begin{enumerate}
  \item Use fuzzy logic to determine the degree to which each rule's conclusion applies based on the input variables.
  \item Combine the rule conclusions to obtain a fuzzy output distribution.
  \item Defuzzify the fuzzy output distribution to obtain a crisp value for Heater Intensity.
\end{enumerate}


\end{enumerate}

\printbibliography

\end{document}