% linear_example_2D.tex -------------------------------------------------
\documentclass[12pt]{article}

\usepackage[utf8]{inputenc}
\usepackage[T1]{fontenc}
\usepackage[brazil]{babel}
\usepackage{lmodern}
\usepackage{amsmath, amssymb, amsfonts, bm}
\usepackage{booktabs}
\usepackage{geometry}
\geometry{a4paper, margin=2.5cm}

\title{Regressão Linear Múltipla (2 Atributos) \\
Resolução Analítica por Mínimos Quadrados}
\author{}
\date{}

\begin{document}
\maketitle

\section{Dados}

\begin{center}
\begin{tabular}{cccc}
\toprule
$i$ & $x_{1i}$ & $x_{2i}$ & $y_i$\\
\midrule
1 & 0 & 0 & 1\\
2 & 1 & 0 & 2\\
3 & 0 & 1 & 2\\
4 & 1 & 1 & 3\\
\bottomrule
\end{tabular}
\end{center}

\bigskip
Modelo com intercepto:
\[
\hat{y}_i = w_0 + w_1 x_{1i} + w_2 x_{2i}
           = \bm{x}_i^\top \bm{w}, 
\qquad
\bm{x}_i =
\begin{bmatrix}
1 \\ x_{1i} \\ x_{2i}
\end{bmatrix},
\;
\bm{w} =
\begin{bmatrix}
w_0 \\ w_1 \\ w_2
\end{bmatrix}.
\]

\section{Formulação matricial}

\[
\bm{y} =
\begin{bmatrix}
1\\2\\2\\3
\end{bmatrix},
\quad
X =
\begin{bmatrix}
1 & 0 & 0\\
1 & 1 & 0\\
1 & 0 & 1\\
1 & 1 & 1
\end{bmatrix}.
\]

A solução de mínimos quadrados é dada pela equação normal
\[
\boxed{\,
\hat{\bm{w}}
 = (X^{\!\top}X)^{-1} X^{\!\top}\bm{y}\,}.
\]

\subsection*{Passo 1: calcular $X^{\!\top}X$ e $X^{\!\top}\bm{y}$}

\[
X^{\!\top}X =
\begin{bmatrix}
4 & 2 & 2\\
2 & 2 & 1\\
2 & 1 & 2
\end{bmatrix},\qquad
X^{\!\top}\bm{y} =
\begin{bmatrix}
8\\5\\5
\end{bmatrix}.
\]

\subsection*{Passo 2: resolver $(X^{\!\top}X)\bm{w}=X^{\!\top}\bm{y}$}

\[
\begin{cases}
4w_0 + 2w_1 + 2w_2 = 8 \\
2w_0 + 2w_1 +   w_2 = 5 \\
2w_0 +   w_1 + 2w_2 = 5
\end{cases}
\]

Eliminação de Gauss:

\[
\begin{aligned}
\text{(2)}\times 2 &: 4w_0 + 4w_1 + 2w_2 = 10 \\
\text{(2)}\times 2 - (1) &: 2w_1 = 2 \;\Rightarrow\; w_1 = 1 \\[6pt]
\text{(3)}\times 2 &: 4w_0 + 2w_1 + 4w_2 = 10 \\
\text{(3)}\times 2 - (1) &: 2w_2 = 2 \;\Rightarrow\; w_2 = 1 \\[6pt]
\text{subst. em (2)} &: 2w_0 + 2(1) + 1 = 5 \\
&\Rightarrow 2w_0 = 2 \;\Rightarrow\; w_0 = 1
\end{aligned}
\]

Portanto
\[
\boxed{\;
\hat{\bm{w}} =
\begin{bmatrix}
1 \\ 1 \\ 1
\end{bmatrix}}
\]

\section{Modelo ajustado}

\[
\hat{y} \;=\; 1 + 1\cdot x_1 + 1\cdot x_2
\;=\; 1 + x_1 + x_2.
\]

Todos os quatro pontos são ajustados \emph{exatamente} (erro zero), pois os dados alinham-se perfeitamente com esse plano linear.

\section{Observações finais}

\begin{itemize}
  \item A matriz $X^{\!\top}X$ é simétrica e definida positiva quando as colunas de $X$ são linearmente independentes.
  \item Para conjuntos maiores, usa-se tipicamente fatoração de Cholesky ou QR em vez de inversão explícita, por estabilidade numérica.
  \item O coeficiente de cada atributo vale 1: cada aumento unitário em $x_1$ ou $x_2$ incrementa $\hat{y}$ em 1.
\end{itemize}

% ----------------------------------------------------------------------
% Métodos adicionais para computar $(X^{\!\top}X)^{-1}$ manualmente
% ----------------------------------------------------------------------

\section{Método Gauss--Jordan para $A^{-1}$}
Considere $A = X^{\!\top}X = \begin{bmatrix}4 & 2 & 2\\2 & 2 & 1\\2 & 1 & 2\end{bmatrix}$.  Monte a matriz aumentada $[A\,|\,I_3]$ e aplique operações elementares de linha até obter $[I_3\,|\,A^{-1}]$.

\[
\left[
\begin{array}{ccc|ccc}
4 & 2 & 2 & 1 & 0 & 0\\
2 & 2 & 1 & 0 & 1 & 0\\
2 & 1 & 2 & 0 & 0 & 1
\end{array}
\right]
\xrightarrow{\text{Gauss--Jordan}}
\left[
\begin{array}{ccc|ccc}
1 & 0 & 0 & \tfrac34 & -\tfrac34 & 0\\
0 & 1 & 0 & -\tfrac12 & 1 & 0\\
0 & 0 & 1 & -\tfrac12 & 0 & 1
\end{array}
\right].
\]

Portanto
\[
A^{-1} =
\begin{bmatrix}
\tfrac34 & -\tfrac34 & 0\\[4pt]
-\tfrac12 & 1 & 0\\[4pt]
-\tfrac12 & 0 & 1
\end{bmatrix}.
\]

\section{Método da Matriz Adjunta (Cofatores)}
O determinante de $A$ vale $\det A = 4$.  Os cofatores $C_{ij}=(-1)^{i+j}\det M_{ij}$ resultam na matriz de cofatores
\[
C =
\begin{bmatrix}
\phantom{-}3 & -2 & -2\\
-3 & \phantom{-}4 &  0\\
\phantom{-}0 & \phantom{-}0 &  4
\end{bmatrix}.
\]
A adjunta é $\operatorname{adj}(A)=C^{\!\top}$, logo
\[
A^{-1}=\frac{1}{\det A}\,\operatorname{adj}(A)=\frac14
\begin{bmatrix}
3 & -3 & 0\\
-2 & 4 & 0\\
-2 & 0 & 4
\end{bmatrix}
=
\begin{bmatrix}
0.75 & -0.75 & 0\\
-0.5 & 1 & 0\\
-0.5 & 0 & 1
\end{bmatrix}.
\]

\subsection*{Aplicando a fórmula dos pesos}
Teste os dois métodos multiplicando $(X^{\!\top}X)^{-1}X^{\!\top}\bm{y}$:\[
\hat{\bm w} =
\begin{bmatrix}
1\\1\\1
\end{bmatrix},
\]confirmando o resultado obtido via sistema linear.

\end{document}
% ----------------------------------------------------------------------
