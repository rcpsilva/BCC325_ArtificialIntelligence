\documentclass[12pt]{article}
\usepackage[utf8]{inputenc}
\usepackage[T1]{fontenc}
\usepackage[brazil]{babel}
\usepackage{amsmath,amssymb,booktabs}
\usepackage{geometry}
\geometry{a4paper, margin=2.5cm}

\title{BCC740 – Inteligência Artificial \\ Exame Especial Total}
\author{}
\date{}

\begin{document}
\maketitle

\vspace{-0.5cm}

\noindent \textbf{Instruções:}  
Cada questão apresenta uma resposta incorreta ou parcialmente incorreta. Sua tarefa é analisar e explicar por que a resposta está equivocada.  
Você não deve apenas indicar a resposta correta; é necessário justificar claramente os erros.

\begin{enumerate}

\item No contexto de busca em espaços de estados, o que caracteriza um problema de busca e quais são seus componentes fundamentais?

\textbf{Resposta incorreta:}  
“Um problema de busca é qualquer problema que envolve encontrar uma resposta. Seus componentes são apenas o estado inicial e o objetivo final.”  

Explique por que essa resposta está incorreta ou parcialmente incorreta.

\item Relacione espaços de estados e grafos. Descreva os elementos de um problema de busca representado como grafo.

\textbf{Resposta incorreta:}  
“O espaço de estados não pode ser representado como grafo, pois estados não possuem conexões. No grafo só importa o nó inicial e o final.”  

Explique por que essa resposta está incorreta ou parcialmente incorreta.

\item Apresente o algoritmo genérico de busca e explique como ele pode ser adaptado para busca em profundidade e busca em largura.

\textbf{Resposta incorreta:}  
“O algoritmo genérico de busca sempre usa fila. Para profundidade basta continuar usando fila e para largura basta inverter a ordem de inserção.”  

Explique por que essa resposta está incorreta ou parcialmente incorreta.

\item Em um CSP com variáveis \(A,B,C \in \{1,2,3,4\}\) e restrições \(A<B\), \(B<C\), quantas atribuições satisfazem todas as restrições?

\textbf{Resposta incorreta:}  
“Todas as 64 atribuições satisfazem as restrições, já que o domínio é de tamanho 4.”  

Explique por que essa resposta está incorreta ou parcialmente incorreta.

\item Classifique cada uma das seguintes restrições quanto à aridade e forma de representação:  
(a) \(A \neq B\), (b) \(\text{Sala}(\text{Aula1}) = \text{Sala2}\),  
(c) tabela com combinações válidas de três variáveis, (d) \(E<A \land E<B \land E<C \land E<D\).  

\textbf{Resposta incorreta:}  
“Todas são unárias e todas são representadas de forma intencional.”  

Explique por que essa resposta está incorreta ou parcialmente incorreta.

\item Para o modelo linear múltiplo com dados simples de entrada \(x_1,x_2\) e saída \(y\), um aluno montou a matriz \(X\) como:
\[
X = \begin{bmatrix}
0 & 0 \\
1 & 0 \\
0 & 1 \\
1 & 1
\end{bmatrix}
\]
sem incluir coluna de 1’s.  

Explique por que essa construção está incorreta ou parcialmente incorreta.

\item Considere a regressão logística para classificar e-mails como spam (\(y=1\)) ou não spam (\(y=0\)). Para uma entrada, o modelo retorna \(\hat{y} = 0.9\). Como interpretar?

\textbf{Resposta incorreta:}  
“Isso significa que o modelo está 100\% certo de que o e-mail é spam.”  

Explique por que essa resposta está incorreta ou parcialmente incorreta.

\item Na figura abaixo, duas curvas de aprendizado representam comportamentos distintos:  

\textbf{Resposta incorreta:}  
“A curva (a) representa overfitting e a curva (b) representa underfitting.”  

Explique por que essa resposta está incorreta ou parcialmente incorreta.

\item Quais funções de perda podem ser usadas para treinar uma rede neural em problemas de regressão e em problemas de classificação?

\textbf{Resposta incorreta:}  
“Em ambos os casos sempre usamos a descida do gradiente como função de perda.”  

Explique por que essa resposta está incorreta ou parcialmente incorreta.

\item Em um conjunto de dados altamente desbalanceado (95\% de negativos, 5\% de positivos), um colega afirmou que a acurácia é suficiente para avaliar o modelo.

\textbf{Resposta incorreta:}  
“Sim, porque se o modelo classificar tudo como negativo ele já terá 95\% de acurácia, o que mostra ótimo desempenho.”  

Explique por que essa resposta está incorreta ou parcialmente incorreta.

\end{enumerate}

\section*{Rubrica de Avaliação}
Cada questão será avaliada segundo os seguintes critérios:

\begin{center}
\begin{tabular}{p{4.5cm}p{10cm}}
\toprule
\textbf{Critério} & \textbf{Descrição}  \\
\midrule
Identificação do erro & Reconhece corretamente o(s) ponto(s) incorreto(s) ou incompletos da resposta \\
Explicação conceitual & Explica por que o ponto identificado está incorreto, demonstrando compreensão teórica \\
Clareza e precisão & Explicação clara, sem ambiguidades, com terminologia adequada \\
Profundidade adicional & Complementa com a resposta correta, exemplo ou contextualização \\
\midrule
\bottomrule
\end{tabular}
\end{center}

\end{document}
