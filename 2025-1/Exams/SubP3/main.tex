% exam_regression_incorrect.tex -----------------------------------------
\documentclass[12pt]{article}

\usepackage[utf8]{inputenc}
\usepackage[T1]{fontenc}
\usepackage[brazil]{babel}
\usepackage{lmodern}
\usepackage{amsmath, amssymb, amsfonts, bm}
\usepackage{booktabs}
\usepackage{geometry}
\geometry{a4paper, margin=2.5cm}

\title{BCC740 - Inteligência Artificial \\ Exame Especial Parcial - Prova 3}
\author{}
\date{}

\begin{document}
\maketitle

\vspace{-1.5cm}

*Verifique se a sua folha de questões possui 8 questões.*

\begin{enumerate}
    \item Para o modelo linear múltiplo (modelo linear com múltiplas variáveis de entrada), considere o seguinte conjunto de dados:
    \[
    \begin{array}{ccc}
      x_1 & x_2 & y \\
      \hline
      0 & 0 & 2 \\
      1 & 0 & 4 \\
      0 & 1 & 3 \\
      1 & 1 & 7 \\
    \end{array}
    \]
    Monte a matriz \(X\) e o vetor \(\bm{y}\).

    \textbf{Resposta incorreta:}
    \[
    X = \begin{bmatrix}
    0 & 0 \\
    1 & 0 \\
    0 & 1 \\
    1 & 1
    \end{bmatrix}, \quad
    \bm{y} = \begin{bmatrix}2 \\ 4 \\ 3 \\ 7\end{bmatrix}
    \]

    Explique por que essa resposta está incorreta ou parcialmente incorreta.

    \item Por que é comum adicionar uma coluna de uns na matriz \(X\)?

    \textbf{Resposta incorreta:}  
    “A coluna de uns em \(X\) não é necessária.”

    Explique por que essa resposta está incorreta ou parcialmente incorreta.


    
    \item Considere a função de verossimilhança
    \[
    L(\bm{w}) = \prod_{i=1}^{n} \hat{y}_i^{y_i}(1-\hat{y}_i)^{1-y_i}.
    \]
    
    Por que ela não é comumente utilizada em problemas de classificação? Qual é a alternativa?  
    \textbf{Resposta incorreta:}  
    “A função de verossimilhança é suficiente para classificação, não é necessário transformá-la.”

    Explique por que essa resposta está incorreta ou parcialmente incorreta.

    \item Considere o modelo linear múltiplo com pesos \(\hat{w}_0 = 2\), \(\hat{w}_1 = 3\), \(\hat{w}_2 = -1\).  
    Explique o significado de cada peso no contexto da regressão.  

    \textbf{Resposta incorreta:}  
    “O peso \(\hat{w}_0 = 2\) significa que a variável resposta \(y\) sempre será 2, independentemente dos valores de \(x_1\) e \(x_2\).  
    O peso \(\hat{w}_1 = 3\) indica que, quando \(x_1\) aumenta em 1, \(y\) aumenta em 3 apenas se \(x_2 = 0\).  
    Já o peso \(\hat{w}_2 = -1\) não tem interpretação clara porque é negativo.”  

    Explique por que essa resposta está incorreta ou parcialmente incorreta.  



    \item Considere um classificador logístico treinado para prever se um e-mail é spam (\(y=1\)) ou não spam (\(y=0\)).  
    Para uma determinada entrada, o modelo retorna \(\hat{y} = 0.8\).  
    Como interpretar esse resultado?  

    \textbf{Resposta incorreta:}  
    “O modelo tem certeza absoluta de que o e-mail é spam, pois \(\hat{y} = 0.8\) significa probabilidade 100\%.”  

    Explique por que essa resposta está incorreta ou parcialmente incorreta.  

    \item Um colega afirmou:  
    “Para qualquer problema de aprendizado supervisionado, seja regressão ou classificação, a função de perda adequada é sempre o erro quadrático médio (MSE), já que ele mede o quanto o modelo erra.”  
    Você concorda com essa afirmação?  

    \textbf{Resposta incorreta:}  
    “Não. O MSE é adequado apenas para problemas de regresão. Para problemas de classificação, a função de perda mais adequada é a acurácia.”  

    Explique por que essa resposta está incorreta ou parcialmente incorreta.  

    \item Em um problema de regressão linear múltipla, um aluno avaliou o desempenho do modelo apenas pela acurácia (proporção de previsões exatas).  
    Você considera essa métrica adequada?  

    \textbf{Resposta incorreta:}  
    “Sim, a acurácia é a melhor métrica porque mostra diretamente quantos valores previstos coincidiram exatamente com os valores reais, o que é o objetivo final do modelo.”  

    Explique por que essa resposta está incorreta ou parcialmente incorreta.  

    \item Considere a regressão logística treinada em um conjunto de dados altamente desbalanceado (95\% de exemplos negativos e 5\% de positivos).  
    Um colega afirmou que a acurácia é suficiente para avaliar o desempenho do modelo.  

    \textbf{Resposta incorreta:}  
    “Sim, pois se o modelo classificar todos os exemplos como negativos, ele terá 95\% de acurácia, o que demonstra ótimo desempenho.”  

    Explique por que essa resposta está incorreta ou parcialmente incorreta.  



\end{enumerate}

\vspace{1cm}
\section*{Rubrica de Avaliação}
Cada questão será avaliada segundo os seguintes critérios:

\begin{center}
\begin{tabular}{p{4.5cm}p{10cm}}
\toprule
\textbf{Critério} & \textbf{Descrição}  \\
\midrule
Identificação do erro & Reconhece corretamente o(s) ponto(s) incorreto(s) ou incompletos da resposta \\
Explicação conceitual & Explica por que o ponto identificado está incorreto, demonstrando compreensão teórica \\
Clareza e precisão & Explicação clara, sem ambiguidades, com terminologia adequada \\
Profundidade adicional & Complementa com a resposta correta, exemplo ou contextualização \\
\midrule
\bottomrule
\end{tabular}
\end{center}

\end{document}
% ----------------------------------------------------------------------
